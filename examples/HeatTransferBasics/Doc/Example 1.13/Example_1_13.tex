
\documentclass{modelica}

\usepackage{textcomp}

\hypersetup{%
	pdftitle  = {EXAMPLE 1-13 Heating of a Plate by Solar Energy},
	pdfauthor = {Javier Bonilla},
        pdfsubject = {Heat and Mass Transfer - A Practical Approach},
        pdfkeywords = {Heat transfer, mass transfer, thermodynamics},
	colorlinks,
	linkcolor=black,
	urlcolor=black,
	citecolor=black,
	pdfpagelayout = SinglePage,
	pdfcreator = pdflatex,
	pdfproducer = pdflatex}

% begin the document
\begin{document}

\thispagestyle{empty}
\date{} % <--- leave date empty

\section*{EXAMPLE 1-13 Heating of a Plate by Solar Energy}

\subsection*{Modelica code}


\begin{lstlisting}[mathescape=true] 
model Example_1_13 "Heating of a Plate by Solar Energy"
  import Modelica.SIunits;
  import Modelica.SIunits.Conversions.NonSIunits;

  parameter SIunits.HeatFlux                  Irr(min=0) =           700  "Solar";
  parameter Real                              alpha(min=0,max=1) =   0.6  "Abs.  (-)";
  parameter SIunits.CoefficientOfHeatTransfer h(min=0) =             50   "(W/(m^2 K))";
  parameter NonSIunits.Temperature_degC       T_surr(min=-273.15) =  25   "Surr. T (C)";

  output SIunits.EnergyFlowRate      E_gain "Energy gained (J/m^2)";
  output SIunits.EnergyFlowRate      E_lost "Energy lost (J/m^2)";
  output NonSIunits.Temperature_degC T_s    "Plate surface temperature (C)";

equation 
  E_gain = alpha*Irr;
  E_lost = h*(T_s-T_surr);
  E_gain = E_lost;
  
end Example_1_13;  
\end{lstlisting}

\end{document}
