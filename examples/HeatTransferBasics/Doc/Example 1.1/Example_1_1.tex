
\documentclass{modelica}

\usepackage{textcomp}

\hypersetup{%
	pdftitle  = {EXAMPLE 1-1 Heating of a Copper Ball},
	pdfauthor = {Javier Bonilla},
        pdfsubject = {Heat and Mass Transfer - A Practical Approach},
        pdfkeywords = {Heat transfer, mass transfer, thermodynamics},
	colorlinks,
	linkcolor=black,
	urlcolor=black,
	citecolor=black,
	pdfpagelayout = SinglePage,
	pdfcreator = pdflatex,
	pdfproducer = pdflatex}

% begin the document
\begin{document}

\thispagestyle{empty}
\date{} % <--- leave date empty

\section*{EXAMPLE 1-1 Heating of a Copper Ball}

\subsection*{Modelica code}


\begin{lstlisting}[mathescape=true] 
model Example_1_1 "Heating a copper ball"

  import Modelica.Constants;
  import Modelica.SIunits;
  import Modelica.SIunits.Conversions.NonSIunits;

  parameter SIunits.Density              rho =             8950 "Density (kg/m3)";
  parameter SIunits.Diameter             D =               0.1  "Diameter (m)";
  parameter SIunits.SpecificHeatCapacity cp(min=0) =       395  "Heat cap. (J/(kg.C))";
  parameter NonSIunits.Temperature_degC  T1(min=-273.15) = 100  "Initial temp. (C)";
  parameter NonSIunits.Temperature_degC  T2(min=-273.15) = 150  "Final temp. (C)";
  parameter SIunits.Time                 dT(min=1e-9) =    1800 "Time interval (s)";

  output SIunits.Area         A    "Area (m2)";
  output SIunits.Volume       V    "Volume (m3)";
  output SIunits.Mass         m    "Mass (kg)";
  output SIunits.Heat         Q    "Heat transferred (J)";
  output SIunits.HeatFlowRate Qavg "Heat flow rate (W)";
  output SIunits.HeatFlux     qavg "Heat flux (W/m2)";

equation 

  A    = Constants.pi*D^2;
  V    = A*D/6;
  m    = rho*V;
  Q    = m*cp*(T2-T1);
  Q    = Qavg*dT;
  Qavg = qavg*A;
  
end Example_1_1;  
\end{lstlisting}

\end{document}
