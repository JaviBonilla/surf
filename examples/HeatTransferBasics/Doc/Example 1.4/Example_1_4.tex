
\documentclass{modelica}

\usepackage{textcomp}

\hypersetup{%
	pdftitle  = {EXAMPLE 1-4 Electric Heating of a House at High Elevation},
	pdfauthor = {Javier Bonilla},
        pdfsubject = {Heat and Mass Transfer - A Practical Approach},
        pdfkeywords = {Heat transfer, mass transfer, thermodynamics},
	colorlinks,
	linkcolor=black,
	urlcolor=black,
	citecolor=black,
	pdfpagelayout = SinglePage,
	pdfcreator = pdflatex,
	pdfproducer = pdflatex}

% begin the document
\begin{document}

\thispagestyle{empty}
\date{} % <--- leave date empty

\section*{EXAMPLE 1-4 Electric Heating of a House at High Elevation}

\subsection*{Modelica code}


\begin{lstlisting}[mathescape=true] 
model Example_1_4 "Electric Heating of a House at High Elevation"
  import Modelica.SIunits;
  import Modelica.SIunits.Conversions;
  import Modelica.SIunits.Conversions.NonSIunits;

  constant Real R = 287 "Gas constant ((Pa m^3)/(Kg K))";

  parameter SIunits.Area                                   A(min=0) =        200 ;
  parameter SIunits.Length                                 L(min=0) =        3;
  parameter SIunits.Pressure                               P_atm(min=0) =    84600;
  parameter NonSIunits.Temperature_degC                    T1(min=-273.15) = 10;
  parameter NonSIunits.Temperature_degC                    T2(min=-273.15) = 20;
  parameter SIunits.SpecificHeatCapacityAtConstantPressure cp(min=0) =       1007;
  parameter Real                                           cost_e(min=0) =   0.075;

  output SIunits.SpecificHeatCapacityAtConstantVolume cv;
  output SIunits.Volume                               V       "Volume (m^3)";
  output SIunits.TemperatureDifference                dT      "Temp. difference (C)";
  output SIunits.Mass                                 m       "Mass (Kg)";
  output SIunits.Energy                               E_cte_v "Energy (J)";
  output SIunits.Energy                               E_cte_p "Energy (J)";
  output Real                                         cost_v  "Cost (dollar)";
  output Real                                         cost_p  "Cost (dollar)";

equation 
  V       = A*L;
  m       = P_atm*V/(R*Conversions.from_degC(T1));
  cv      = cp - R;
  dT      = T2-T1;
  E_cte_v = m*cv*dT;
  E_cte_p = m*cp*dT;
  cost_v  = E_cte_v*cost_e/(3600*10^3);
  cost_p  = E_cte_p*cost_e/(3600*10^3);
  
end Example_1_4;  
\end{lstlisting}

\end{document}
