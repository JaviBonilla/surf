
\documentclass{modelica}

\usepackage{textcomp}

\hypersetup{%
	pdftitle  = {EXAMPLE 1-3 Heat Loss from Heating Ducts in a Basement},
	pdfauthor = {Javier Bonilla},
        pdfsubject = {Heat and Mass Transfer - A Practical Approach},
        pdfkeywords = {Heat transfer, mass transfer, thermodynamics},
	colorlinks,
	linkcolor=black,
	urlcolor=black,
	citecolor=black,
	pdfpagelayout = SinglePage,
	pdfcreator = pdflatex,
	pdfproducer = pdflatex}

% begin the document
\begin{document}

\thispagestyle{empty}
\date{} % <--- leave date empty

\section*{EXAMPLE 1-3 Heat Loss from Heating Ducts in a Basement}

\subsection*{Modelica code}


\begin{lstlisting}[mathescape=true] 
model Example_1_3 "Heat Loss from Heating Ducts in a Basement"
  import Modelica.SIunits;
  import Modelica.SIunits.Conversions;
  import Modelica.SIunits.Conversions.NonSIunits;

  constant Real R = 287 "Gas constant ((Pa m^3)/(Kg K))";

  parameter SIunits.Length               len(min=0) =         5           "Lenght (m)";
  parameter SIunits.Length               L1(min=0) =          0.20        "Side1 L. (m)";
  parameter SIunits.Length               L2(min=0) =          0.25        "Side2 L. (m)";
  parameter SIunits.Pressure             p(min=0) =           100*10^3    "Pres. (Pa)";
  parameter NonSIunits.Temperature_degC  T_in(min=-273.15) =  60          "Inlet T (C)";
  parameter NonSIunits.Temperature_degC  T_out(min=-273.15) = 54          "Outlet T (C)";
  parameter SIunits.Velocity             v(min=0) =           5           "Vel. (m/s)";
  parameter SIunits.SpecificHeatCapacity cp(min=0) =          1007        "(J/(Kg K))";
  parameter SIunits.Efficiency           e(min=0,max=1) =     0.80        "Effic. (%)";
  parameter Real                         cost_gas(min=0) =    1.60        "(dol./therm)";
  parameter SIunits.Energy               therm(min=0) =       105500*10^3 "1 therm in J";

  output SIunits.Area                  A     "Area (m^2)";
  output SIunits.Density               rho   "Density (kg/m^3)";
  output SIunits.MassFlowRate          m_dot "Mass flow rate (kg/s)";
  output SIunits.TemperatureDifference dT    "Temperature difference (K)";
  output SIunits.HeatFlowRate          Q     "Heat flow rate (W)";
  output Real                          cost  "Cost (dollar/h)";

equation 
   rho   = p/(R*Conversions.from_degC(T_in));
   A     = L1*L2;
   m_dot = rho*v*A;
   dT    = T_in - T_out;
   Q     = m_dot *cp * dT;
   cost  = (Q*3600*cost_gas)/(e*therm);
  
end Example_1_3;  
\end{lstlisting}

\end{document}
