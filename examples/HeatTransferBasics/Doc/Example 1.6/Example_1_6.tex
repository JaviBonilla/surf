
\documentclass{modelica}

\usepackage{textcomp}

\hypersetup{%
	pdftitle  = {EXAMPLE 1-6 Measuring the Thermal Conductivity of a Material},
	pdfauthor = {Javier Bonilla},
        pdfsubject = {Heat and Mass Transfer - A Practical Approach},
        pdfkeywords = {Heat transfer, mass transfer, thermodynamics},
	colorlinks,
	linkcolor=black,
	urlcolor=black,
	citecolor=black,
	pdfpagelayout = SinglePage,
	pdfcreator = pdflatex,
	pdfproducer = pdflatex}

% begin the document
\begin{document}

\thispagestyle{empty}
\date{} % <--- leave date empty

\section*{EXAMPLE 1-6 Measuring the Thermal Conductivity of a Material}

\subsection*{Modelica code}


\begin{lstlisting}[mathescape=true] 
  import Modelica.Constants;
  import Modelica.SIunits;
  import Modelica.SIunits.Conversions.NonSIunits;

  parameter SIunits.Diameter            D =         0.05 "Diameter (m)";
  parameter SIunits.Length              L(min=0) =  0.03 "Lenght (m)";
  parameter SIunits.Voltage             V(min=0) =  110  "Voltage (V)";
  parameter SIunits.Current             I(min=0) =  0.4  "Current (A)";
  parameter NonSIunits.Temperature_degC dT =        15   "Temperature difference (C)";

  output SIunits.Area                A  "Area (m^2)";
  output SIunits.Power               We "Power (W)";
  output SIunits.HeatFlowRate        Q  "Heat flow rate (W)";
  output SIunits.ThermalConductivity k "Thermal conductivity (W/(m K))";

equation 
  A  = 1/4*Constants.pi*D^2;
  We = V*I;
  Q  = 1/2*We;
  Q  = k*A*dT/L;
  
end Example_1_6;  
\end{lstlisting}

\end{document}
