
\documentclass{modelica}

\usepackage{textcomp}

\hypersetup{%
	pdftitle  = {EXAMPLE 1-8 Measuring Convection Heat Transfer Coefficient},
	pdfauthor = {Javier Bonilla},
        pdfsubject = {Heat and Mass Transfer - A Practical Approach},
        pdfkeywords = {Heat transfer, mass transfer, thermodynamics},
	colorlinks,
	linkcolor=black,
	urlcolor=black,
	citecolor=black,
	pdfpagelayout = SinglePage,
	pdfcreator = pdflatex,
	pdfproducer = pdflatex}

% begin the document
\begin{document}

\thispagestyle{empty}
\date{} % <--- leave date empty

\section*{EXAMPLE 1-8 Measuring Convection Heat Transfer Coefficient}

\subsection*{Modelica code}


\begin{lstlisting}[mathescape=true] 
model Example_1_8 "Measuring Convection Heat Transfer Coefficient"
  import Modelica.Constants;
  import Modelica.SIunits;
  import Modelica.SIunits.Conversions.NonSIunits;

  parameter SIunits.Length              l(min=0) =            0.2  "Lenght (m)";
  parameter SIunits.Diameter            D =                   0.03 "Diameter (m)";
  parameter NonSIunits.Temperature_degC T_surr(min=-273.15) = 15   "Surr. temp. (C)";
  parameter NonSIunits.Temperature_degC T_s(min=-273.15) =    152  "Surface temp. (C)";
  parameter SIunits.Current             I(min=0) =            1.5  "Current (A)";
  parameter SIunits.Voltage             V(min=0) =            60   "Voltage (V)";

  output SIunits.Area                      A_s     "Surface area (m^2)";
  output SIunits.HeatFlowRate              Q_conv  "Convective heat flow rate (W/m^2)";
  output SIunits.CoefficientOfHeatTransfer h       "Convective heat transfer coefficient (W/(m^2 K))";

equation 
  A_s    = Constants.pi*l*D;
  Q_conv = V*I;
  Q_conv = h*A_s*(T_s-T_surr);
  
end Example_1_8;  
\end{lstlisting}

\end{document}
