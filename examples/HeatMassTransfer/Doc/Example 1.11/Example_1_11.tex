
\documentclass{modelica}

\usepackage{textcomp}

\hypersetup{%
	pdftitle  = {EXAMPLE 1-11 Heat Transfer between Two Isothermal Plates},
	pdfauthor = {Javier Bonilla},
        pdfsubject = {Heat and Mass Transfer - A Practical Approach},
        pdfkeywords = {Heat transfer, mass transfer, thermodynamics},
	colorlinks,
	linkcolor=black,
	urlcolor=black,
	citecolor=black,
	pdfpagelayout = SinglePage,
	pdfcreator = pdflatex,
	pdfproducer = pdflatex}

% begin the document
\begin{document}

\thispagestyle{empty}
\date{} % <--- leave date empty

\section*{EXAMPLE 1-11 Heat Transfer between Two Isothermal Plates}

\subsection*{Modelica code}


\begin{lstlisting}[mathescape=true] 
model Example_1_11 "Heat Transfer between Two Isothermal Plates"
  import Modelica.Constants;
  import Modelica.SIunits;

  parameter SIunits.Temperature         T1 =                   300     "Plate 1 T (K)";
  parameter SIunits.Temperature         T2 =                   200     "Plate 2 T (K)";
  parameter SIunits.Length              L(min=0) =             0.01    "Lenght (m)";
  parameter SIunits.Area                A(min=0) =             1       "Area (m^2)";
  parameter Real                        epsilon(min=0,max=1) = 1       "Emissivity (-)";
  parameter SIunits.ThermalConductivity k_air(min=0) =         0.0219  "Air con";
  parameter SIunits.ThermalConductivity k_ure(min=0) =         0.026   "Urethane";
  parameter SIunits.ThermalConductivity k_sup(min=0) =         0.00002 "Superinsulation";

  output SIunits.HeatFlowRate Q_cond_a  "Heat flow rate (W)";
  output SIunits.HeatFlowRate Q_cond_c  "Heat flow rate (W)";
  output SIunits.HeatFlowRate Q_cond_d  "Heat flow rate (W)";
  output SIunits.HeatFlowRate Q_rad     "Heat flow rate (W)";
  output SIunits.HeatFlowRate Q_a       "Heat flow rate (W)";
  output SIunits.HeatFlowRate Q_b       "Heat flow rate (W)";
  output SIunits.HeatFlowRate Q_c       "Heat flow rate (W)";
  output SIunits.HeatFlowRate Q_d       "Heat flow rate (W)";

  Real Q_cond_par;

equation 
  Q_cond_par = A*(T1-T2)/L;
  Q_cond_a   = k_air*Q_cond_par;
  Q_cond_c   = k_ure*Q_cond_par;
  Q_cond_d   = k_sup*Q_cond_par;
  Q_rad      = epsilon*Constants.sigma*A*(T1^4-T2^4);
  Q_a        = Q_cond_a + Q_rad;
  Q_b        = Q_rad;
  Q_c        = Q_cond_c;
  Q_d        = Q_cond_d;
  
end Example_1_11;  
\end{lstlisting}

\end{document}
