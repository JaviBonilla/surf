
\documentclass{modelica}

\usepackage{textcomp}

\hypersetup{%
	pdftitle  = {EXAMPLE 1-10 Heat Loss from a Person},
	pdfauthor = {Javier Bonilla},
        pdfsubject = {Heat and Mass Transfer - A Practical Approach},
        pdfkeywords = {Heat transfer, mass transfer, thermodynamics},
	colorlinks,
	linkcolor=black,
	urlcolor=black,
	citecolor=black,
	pdfpagelayout = SinglePage,
	pdfcreator = pdflatex,
	pdfproducer = pdflatex}

% begin the document
\begin{document}

\thispagestyle{empty}
\date{} % <--- leave date empty

\section*{EXAMPLE 1-10 Heat Loss from a Person}

\subsection*{Modelica code}


\begin{lstlisting}[mathescape=true] 
model Example_1_10 " Heat Loss from a Person"
  import Modelica.Constants;
  import Modelica.SIunits;
  import Modelica.SIunits.Conversions;
  import Modelica.SIunits.Conversions.NonSIunits;

  parameter Real                              epsilon(min=0,max=1) = 0.95 "Emiss. (-)";
  parameter SIunits.CoefficientOfHeatTransfer h(min=0) =             6    "HTC (W/(m^2 K))";
  parameter SIunits.Area                      A_s(min=0) =           1.6  "Area (m^2)";
  parameter NonSIunits.Temperature_degC       T_surr(min=-273.15) =  20   "Surr. T (C)";
  parameter NonSIunits.Temperature_degC       T_s(min=-273.15) =     29   "Surface T (C)";

  output SIunits.HeatFlowRate Q_rad  "Radiation heat losses (W)";
  output SIunits.HeatFlowRate Q_conv "Convection heat losses (W)";
  output SIunits.HeatFlowRate Q_cond "Conduction heat losses (W)";
  output SIunits.HeatFlowRate Q_t    "Total heat losses (W)";

  SIunits.Temperature T_s_K =    Conversions.from_degC(T_s);
  SIunits.Temperature T_surr_K = Conversions.from_degC(T_surr);

equation 
  Q_t    = Q_rad + Q_conv + Q_cond;
  Q_rad  = epsilon*Constants.sigma*A_s*(T_s_K^4-T_surr_K^4);
  Q_conv = h*A_s*(T_s-T_surr);
  Q_cond = 0;
  
end Example_1_10;  
\end{lstlisting}

\end{document}
